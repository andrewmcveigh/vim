\documentclass{article}
\usepackage{graphicx}
\usepackage{graphics}
\usepackage{listings}
\usepackage{hyperref}

\begin{document}
\section{Infrastructure}
\subsection{Level 0}
\includegraphics[width=1\linewidth]{infrastructure}
\subsection{Level 1: Interfaces}
\includegraphics[width=1\linewidth]{level_1-interfaces}
\subsection{Level 2: Models}
\includegraphics[clip,trim=4mm 0mm 0mm 0mm,width=1.1\linewidth]{dbml}
\subsection{Technology Outline}
The package contains the following projects/folders:
\subsubsection{\_database}
Version controlled backup of the current database layout
\subsubsection{\_docs}
Documentation
\subsubsection{\_hook-scripts}
Scripts run on various version control hook events
\subsubsection{HelpdeskService}
The service that checks the mailbox of the Helpdesk user, and monitors the Helpdesk telephone numbers for incoming calls.
\subsubsection{InstallHelpdesk}
The installer for the HelpdeskService.
\subsubsection{MailService.Test}
Defunct
\subsubsection{Messaging}
WCF Web Service used for sending email and writing to event logs.
\subsubsection{Messaging.Test}
Unit tests for Messaging
\subsubsection{Models}
Data models, Linq to SQL Classes, Data repository, Helpers and Reports.
\subsubsection{Models.Test}
Unit tests for Models
\subsubsection{NetworkServiceLibrary}
Library used for checking emails and phone calls from Exchange Web Services and TAPI.
\subsubsection{NetworkServiceLibrary.Test}
Unit tests for NetworkServiceLibrary
\subsubsection{Output}
Output folder for library DLLs to quicken compilation speeds (not sure that it works though!).
\subsubsection{Resources}
Folder containing 3rd party libraries
\subsubsection{ScheduleTimer}
Defunct
\subsubsection{TapiExchange}
WCF Web service used for making TAPI calls.
\subsubsection{TapiWrapper}
Defunct
\subsubsection{TestResults}
Default folder for unit test results.
\subsubsection{TestUtils}
Useful classes and helpers for writing tests.
\subsubsection{web}
The ASP.NET MVC 2 web interface to the helpdesk.
\subsubsection{web.Tests}
Unit tests for web.
\section{Exchange Web Services}
\subsection{Impersonation}
If the mail server is new, has been changed, or updated, the Exchange user needs a special permission setting.\\
\\*First the Service Nodes need to be shown:\\
\indent AD Sites \& Services $\rightarrow$ View $\rightarrow$ Show Service Nodes\\
\\*Then, the setting is found at:\\
\indent AD Sites \& Services
\\*\indent\indent $\rightarrow$ Services
\\*\indent\indent $\rightarrow$ Microsoft Exchange
\\*\indent\indent $\rightarrow$ DOMAIN\_NAME
\\*\indent\indent $\rightarrow$ Administrative Groups
\\*\indent\indent $\rightarrow$ Exchange Administrative Group
\\*\indent\indent $\rightarrow$ Servers
\\*\indent\indent $\rightarrow$ NEW\_MAIL\_SERVER\_NAME
\\*\indent\indent $\rightarrow$ ``right click''
\\*\indent\indent $\rightarrow$ Properties
\\*\indent\indent $\rightarrow$ Security
\\*\indent\indent $\rightarrow$ Advanced
\\*\indent\indent $\rightarrow$ Add
\\*\indent\indent $\rightarrow$ ``find desired user''
\\*\indent\indent $\rightarrow$ ``near bottom of list'' Exchange Web Services Impersonation

\section{Deployment}
Deployment occurs automatically\footnote{The server hosting the HelpdeskService MUST NOT have the ``Services.msc'' dialog running when the repository is deployed.} when the MASTER mercurial branch is pushed to, using the ``incoming'' hook. The version control server ``versions'' handles the deployment.\\
\subsection{Deployment Process}
The deployment process has three stages:
\begin{itemize}
  \item Tracks Commits
  \item Deploys Web
  \item Deploys Service
\end{itemize}
It is called with the ``incoming'' hook calling the script ``track\_and\_deploy.sh''. This script in turn calls the ``deploy.sh'' script, which calls ``deploy\_web.sh'' followed by ``deploy\_service.sh''.
\subsubsection{Track Commits}
\label{sec:trackerintegration}
The Helpdesk is feature/bug/chore tracked on ``pivotaltracker.com''. Upon pushing to the mercurial repository hosted on ``versions'', a command is sent to ``pivotaltracker.com'' to update the project automatically. Certain syntax must be used in the commit message for this to work. The syntax specification can be found at https://www.pivotaltracker.com/help/api?version=v3\\\#scm\_post\_commit\_message\_syntax.
\subsubsection{Deploy Web}
\begin{enumerate}
  \item Mounts the //web.interel.local/sites/helpdesk folder
  \item Clears the temp folder used if present
  \item Runs ``hg  archive -r tip -I 'web/*' /tmp/helpdesk''.
    This exports the ``web'' directory from the ``tip'' of the
    repository to ``/tmp/helpdesk''.
  \item Runs ``sed'' against the ``Web.config'' file to remove
    settings relating to the ``debugging database''.
  \item ``Rsyncs'' the ``/tmp/helpdesk'' with the mounted folder.
  \item Unmounts the folder.
\end{enumerate}
\subsubsection{Deploy Service}
\begin{enumerate}
  \item Mounts the //web.interel.local/services folder
  \item Stops the HelpdeskService on the server, remotely using 
    ``winexe/winexec''.
  \item Uninstalls the HelpdeskService\footnote{This is the point
    that the ``Services.msc'' dialog must be closed at,
    otherwise the service will not uninstall properly.} using 
    ``winexe''.
  \item Waits a few seconds for the service to uninstall, then
    attempts to delete the service's files. It will usually
    fail to delete all the files on the first attempt, so a
    second wait, followed by a second delete command are
    issued.
  \item The files are cleared, exported and sed-ed, as per the
    method used for the Web deployment.
  \item The files are then copied over to the mounted folder
    using ``cp''.
  \item The service is re-installed.
  \item The service is then re-started.
  \item The folder is unmounted.
\end{enumerate}
\subsection{Committing Without Deploying}
Local mercurial commits are recommended for minor unstable versions, merging back to the master branch when they become stable. It is also recommended that the cloned copies are kept in a synchronised service such as ``Dropbox'' for backup and availability purposes.

\section{Web/UI}
\subsection{Markup/HTML}
The web/UI is employing the default view-engine of ASP.NET MVC 2 to generate markup. Using strict views when possible, passing a model to a view and rendering a view based on the model passed. Using the philosophy of ``fat models, skinny views'', the views are left purposefully sparse and minimal.
\subsection{Cascading Style Sheets}
To assist in writing better and quicker CSS, we are using ``SASS'', the CSS extension. The current version of SASS is version 3, although the majority of files in this project are still using SASS 2 syntax. There is no need to update the syntax until a section is updated as the SASS 2 syntax will still compile with a SASS 3 compiler.
\subsection{Scripting/JavaScript}
The JavaScript in this project is written using the jQuery Library. We also employ a custom structural ``framework'' to aid the writing and organisation of JavaScript files, and to help maintain the JavaScript object model. The lightweight ``compiler'' reads a directory of scripts, builds/combines, and optionally minimizes a directory tree of JavaScript files.\\
\\*The JavaScript object model created by the ``compiler'' will resemble the folder structure, and filenames of the directory tree it is built from.\\
\\*For instance:
\lstset{
  literate={~} {$\sim$}{1}, % set tilde as a literal (no process)
  basicstyle=\ttfamily      % Code font
}
\begin{lstlisting}
  |~Scripts
    |~Controllers
      |-!Controller.js
      |-InventoryController.js
      |-TicketsController.js
    |+DOM
    |+Models
\end{lstlisting}
Would be compiled (not minimised) to something like: % Should use minted/pygments for syntax highlighting?
\begin{lstlisting}
  (function($) { 
    $.Scripts = new function() {
      this.Controllers = new function() {
        this._Controller = function() {
        };
        this.InventoryController = function() {
        };
        this.TicketsController = function() {
        };
      };
      this.DOM = new function() {
      };
      this.Models = new function() {
      };
    };
  })
\end{lstlisting}
Which is loaded, by the browser, into the JavaScript object model:
\begin{lstlisting}
  |~$: function(selector, context) // The jQuery object
    |~Scripts: Object
      |~Controllers: Object
        |+TicketsController: Object
        |+InventoryController: Object
        |+_Controller: function()
      |+DOM: Object
      |+Models: Object
\end{lstlisting}
    

\subsection{Server Side}
The server side code of the ``Web'' project is using the following technology:
\begin{itemize}
  \item ASP.NET MVC 2
  \item Linq
  \item Linq to SQL (employed in the ``Models'' library)
  \item C\# 3.0
  \item Microsoft Reports
\end{itemize}

\section{Testing}
The project is currently using the Microsoft Testing Framework built into Visual Studio. We are considering to move to the NUnit Testing Framework, as:
\begin{itemize}
  \item There is more support
  \item It is open source
  \item It seems to run much faster (major point)
\end{itemize}
It is estimated that over half of the testable\footnote{Code that is not directly linked to markup generation, or .NET framework code.} code is currently covered by unit tests.

\section{Project Feature, Chore \& Bug Tracking}
The project uses the feature/chore/bug tracking of the ``Software as a Service'' (SAAS) product ``Pivotal Tracker'' (\url{http://www.pivotaltracker.com/}). The tracker is an ``agile project management tool that enables real time collaboration around a shared, prioritized backlog''. The tracker is also tied in to the projects version-control/deployment system. The integration is outlined in Section \ref{sec:trackerintegration}.

\end{document}
